\documentclass{article}
\usepackage{enumitem}
\usepackage{fullpage}
\renewcommand{\familydefault}{\sfdefault}

\usepackage[dvipsnames]{xcolor}
\usepackage[scaled=1]{helvet}
\usepackage[helvet]{sfmath}
\usepackage[spanish, es-tabla]{babel}
\everymath={\sf}
\usepackage{parskip}
\usepackage[colorinlistoftodos]{todonotes}
\usepackage[colorlinks=true, allcolors=BlueViolet]{hyperref}
\usepackage{graphicx}
\usepackage{titlesec}
\usepackage{fancyhdr}
\usepackage{array}
\usepackage{caption}
\usepackage{tikz}
\usetikzlibrary{trees}

\titleformat{\section}
  {\Large\bfseries\color{BlueViolet}}{\thesection}{1em}{}
\titleformat{\subsection}
  {\large\bfseries\color{BlueViolet}}{\thesubsection}{1em}{}

\setlength{\headheight}{45pt}
\setlength{\headsep}{20pt}
\pagestyle{fancy}
\fancyhf{}
\lhead{\includegraphics[height=40pt]{Images/Logo Uniautonoma.png}}
\chead{\textsf{Universidad Autónoma del Cauca}}
\rhead{\textsf{Torres de Hanói}}
\rfoot{\textsf{\thepage}}

\begin{document}

\begin{titlepage}
\centering
\includegraphics[width=0.4\textwidth]{Images/Logo Uniautonoma.png}\par
\vspace{1cm}
{\bfseries\LARGE Universidad Autónoma del Cauca\par}
\vspace{0.5cm}
{\scshape\Large Taller de Desarrollo de Aplicaciones Web\par}
\vspace{0.5cm}
{\scshape\Huge Torres de Hanói Interactiva\par}
\vspace{1cm}
{\itshape\LARGE HTML, JavaScript y Árbol Binario\par}
\vspace{1.5cm}
{\Large \textbf{Autores:} \\ Daniel Rivas Agredo \\ Thomas Montoya Magon \par}
\vspace{0.5cm}
{\Large \textbf{Docente:} \\ Diego Fernando Prado Osorio\par}
\vfill
{\Large \today \par}
\end{titlepage}

\tableofcontents
\clearpage

\section{Introducción}
La Torre de Hanói es un clásico juego de lógica que consiste en trasladar una serie de discos de una torre a otra, siguiendo ciertas reglas. Aunque su apariencia es sencilla, su resolución implica el uso de un algoritmo recursivo que puede representarse como un árbol binario.

\section{Objetivos}
\begin{itemize}
    \item Implementar el juego de la Torre de Hanói con una interfaz gráfica web interactiva.
    \item Representar la solución del juego mediante un árbol binario.
    \item Analizar la complejidad computacional del algoritmo.
    \item Brindar al usuario una experiencia lúdica y educativa.
\end{itemize}

\section{Descripción General del Juego}
El juego permite al usuario seleccionar el número de discos (de 3 a 8), realizar movimientos manuales, ver una solución automática paso a paso, y observar el árbol binario generado por el algoritmo recursivo. También se muestran estadísticas como el tiempo transcurrido, número de movimientos y el mínimo teórico posible.

\section{Implementación Técnica}

\subsection{Tecnologías Utilizadas}
\begin{itemize}
    \item \textbf{HTML/CSS/JS}: Para el diseño visual e interactividad.
    \item \textbf{Bootstrap y AOS}: Para animaciones y estilos responsivos.
    \item \textbf{JavaScript}: Lógica del juego, validación de reglas y árbol recursivo.
\end{itemize}

\subsection{Estructura del Proyecto}
El proyecto está organizado con la siguiente estructura de archivos y directorios:

\textbf{Torres\_Hanoi/}
\begin{itemize}
    \item \textbf{Archivos principales}:
    \begin{itemize}
        \item \texttt{index.html}: Página principal con el menú de inicio del juego
        \item \texttt{juego.html}: Interfaz principal del juego de Torres de Hanói
        \item \texttt{informe\_torres\_hanoi.tex}: Documentación técnica del proyecto en LaTeX
    \end{itemize}
    
    \item \textbf{css/}: Directorio de hojas de estilo
    \begin{itemize}
        \item \texttt{menu.css}: Estilos CSS para la página del menú principal
        \item \texttt{juego.css}: Estilos CSS específicos para la interfaz del juego
    \end{itemize}
    
    \item \textbf{js/}: Directorio con la lógica JavaScript
    \begin{itemize}
        \item \texttt{menu.js}: Funcionalidades interactivas del menú principal
        \item \texttt{juego.js}: Lógica completa del juego, algoritmo recursivo, validaciones y generación del árbol binario
    \end{itemize}
    
    \item \textbf{images/}: Directorio de recursos gráficos
    \begin{itemize}
        \item \texttt{fondo.png}: Imagen de fondo para mejorar la estética de la interfaz
    \end{itemize}
    
    \item \textbf{music/}: Directorio de recursos de audio
    \begin{itemize}
        \item \texttt{cancionmenu.mp3}: Música de fondo para el menú principal
        \item \texttt{cancion.mp3}: Música ambiente durante el desarrollo del juego
    \end{itemize}
    
    \item \textbf{.git/}: Directorio de control de versiones Git con historial completo del proyecto
\end{itemize}

\subsection{Arquitectura del Código}
\begin{itemize}
    \item \textbf{HTML}: Define la interfaz del juego, controles, estadísticas e instrucciones.
    \item \textbf{CSS}: Proporciona diseño visual, animaciones y responsividad.
    \item \textbf{JavaScript}: Gestiona la lógica del juego (movimientos, validaciones, solución automática) y genera el árbol binario de resolución.
\end{itemize}

\subsection{Árbol Binario Recursivo}
La solución se basa en una estructura de árbol recursivo, donde cada nodo representa una sub-tarea en el proceso de resolución. Para resolver el problema con $n$ discos, el algoritmo se divide en tres pasos principales:
\begin{enumerate}
    \item Mover $n-1$ discos de la torre origen a la torre auxiliar
    \item Mover el disco más grande de la torre origen a la torre destino
    \item Mover $n-1$ discos de la torre auxiliar a la torre destino
\end{enumerate}

Para el caso específico de 3 discos, el árbol de llamadas recursivas se estructura de la siguiente manera:

\begin{figure}[h]
\centering
\begin{tikzpicture}
  [level distance=2.2cm,
   level 1/.style={sibling distance=7cm},
   level 2/.style={sibling distance=3.5cm},
   level 3/.style={sibling distance=1.8cm},
   every node/.style={text centered, font=\tiny, rectangle, draw, minimum width=2.5cm, minimum height=0.8cm}]
  \node {Hanoi(3, A→C, B)}
    child {node {Hanoi(2, A→B, C)}
      child {node {Hanoi(1, A→C, B)}}
      child {node {Hanoi(1, C→B, A)}}}
    child {node {Hanoi(2, B→C, A)}
      child {node {Hanoi(1, B→A, C)}}
      child {node {Hanoi(1, A→C, B)}}};
\end{tikzpicture}
\caption{Árbol de recursión para Torres de Hanói con 3 discos. Cada nodo representa una llamada recursiva Hanoi(n, origen→destino, auxiliar).}
\end{figure}

Cada nodo del árbol representa una llamada a la función recursiva \texttt{Hanoi(n, origen→destino, auxiliar)}, donde:
\begin{itemize}
    \item \textbf{n}: Número de discos a mover
    \item \textbf{origen→destino}: Movimiento de torre origen a torre destino
    \item \textbf{auxiliar}: Torre auxiliar utilizada en el proceso
\end{itemize}

La recursividad termina cuando $n=1$, momento en el cual se realiza un movimiento directo del disco. El árbol completo para 3 discos genera exactamente $2^3 - 1 = 7$ movimientos.

\section{Análisis de Complejidad}
El algoritmo de resolución tiene una complejidad de $O(2^n)$, donde $n$ es el número de discos. Esto se debe a que cada disco genera dos subproblemas recursivos de tamaño $n-1$.

\begin{center}
    \begin{tabular}{|c|c|c|}
        \hline
        \textbf{Discos (n)} & \textbf{Movimientos mínimos} & \textbf{Altura del árbol} \\
        \hline
        3 & 7  & 3 \\
        4 & 15 & 4 \\
        5 & 31 & 5 \\
        \hline
    \end{tabular}
\end{center}

\section{Reflexión sobre el uso de estructuras de datos}
La representación del problema como un árbol binario no solo permite visualizar la recursión de manera clara, sino que también ayuda a comprender la lógica de división del problema en subproblemas más pequeños. Este enfoque demuestra cómo las estructuras de datos pueden facilitar el diseño y análisis de algoritmos complejos, y refuerza la importancia de elegir estructuras adecuadas según el contexto. También se identifican límites en la escalabilidad, ya que para valores grandes de $n$, la profundidad del árbol y la cantidad de nodos se vuelve impráctica de manejar sin técnicas adicionales.

\section{Conclusión}
Este proyecto permitió aplicar conceptos fundamentales de programación web junto con estructuras de datos como árboles binarios. La visualización del árbol de llamadas y la solución paso a paso son elementos clave que enriquecen la experiencia del usuario y refuerzan el aprendizaje algorítmico. Además, este trabajo evidencia la utilidad de los árboles binarios en la representación de procesos recursivos y en el análisis de su complejidad.

\end{document}
